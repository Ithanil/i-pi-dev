\subsection{SIMULATION}
\label{SIMULATION}
This is the top level class that deals with the running of the simulation, including holding the simulation specific properties such as the time step and outputting the data.
\paragraph{Fields}
 \begin{itemize}
\item {\bf \hyperref[FORCES]{force} }:
 Deals with the assigning of jobs to different driver codes, and collecting the data.
\paragraph{Attributes}
 \begin{itemize}
\item {\bf type}:
 Specifies which kind of force object is created
{\\ \bf DEFAULT: }'socket'
{\\ \bf OPTIONS: }'socket'
{\\ \bf DATA TYPE: }str
\end{itemize}
 
\paragraph{Fields}
 \begin{itemize}
\item {\bf interface}:
 Specifies the parameters for the socket interface.
\item {\bf parameters}:
 deprecated dictionary of initialization parameters. May be removed in the future.
{\\ \bf DEFAULT: }\{ \}
{\\ \bf DATA TYPE: }dict
\end{itemize}
 
\item {\bf trajectories}:
 A list of the allowed properties to print out the per-atom or per-bead trajectories of. Allowed values are ['positions', 'velocities', 'forces', 'kinetic\_cv', 'centroid'].
{\\ \bf DEFAULT: }[ ]
{\\ \bf DATA TYPE: }str
\paragraph{Attributes}
 \begin{itemize}
\item {\bf shape}:
 The shape of the array
{\\ \bf DEFAULT: }(0,)
{\\ \bf DATA TYPE: }tuple
\end{itemize}
 
\item {\bf \hyperref[ATOMS]{atoms} }:
 Deals with classical simulations.
\paragraph{Fields}
 \begin{itemize}
\item {\bf q}:
 The positions of the atoms, in the format [x1, y1, z1, x2, \ldots  ]
{\\ \bf DIMENSION: }length
{\\ \bf DEFAULT: }[ ]
{\\ \bf DATA TYPE: }float
\item {\bf p}:
 The momenta of the atoms, in the format [px1, py1, pz1, px2, \ldots  ]
{\\ \bf DIMENSION: }momentum
{\\ \bf DEFAULT: }[ ]
{\\ \bf DATA TYPE: }float
\item {\bf natoms}:
 The number of atoms
{\\ \bf DEFAULT: }0
{\\ \bf DATA TYPE: }int
\item {\bf names}:
 The names of the atoms, in the format [name1, name2, \ldots  ]
{\\ \bf DEFAULT: }[ ]
{\\ \bf DATA TYPE: }str
\item {\bf file\_units}:
 The units in which the lengths in the configuration file are given.
{\\ \bf DEFAULT: }''
{\\ \bf OPTIONS: }'', 'nanometer', 'angstrom', 'atomic\_unit'
{\\ \bf DATA TYPE: }str
\item {\bf from\_file}:
 Gives the name of the file from which the configurations are taken, if present.
{\\ \bf DEFAULT: }''
{\\ \bf DATA TYPE: }str
\item {\bf m}:
 The masses of the atoms, in the format [m1, m2, \ldots  ]
{\\ \bf DIMENSION: }mass
{\\ \bf DEFAULT: }[ ]
{\\ \bf DATA TYPE: }float
\item {\bf init\_temp}:
 The temperature at which the initial velocity distribution is taken, if applicable.
{\\ \bf DIMENSION: }temperature
{\\ \bf DEFAULT: }-1.0
{\\ \bf DATA TYPE: }float
\end{itemize}
 
\item {\bf step}:
 How many time steps have been done.
{\\ \bf DEFAULT: }0
{\\ \bf DATA TYPE: }int
\item {\bf initialize}:
 A dictionary giving the properties of the system that need to be initialized, and their initial values. The allowed keywords are ['velocities']. The initial value of 'velocities' corresponds to the temperature to initialise the velocity distribution from. If 0, then the sysytem temperature is used.
{\\ \bf DEFAULT: }\{ \}
{\\ \bf DATA TYPE: }dict
\item {\bf properties}:
 A list of the properties that will be printed in the properties output file. See the manual for a full list of acceptable names.
{\\ \bf DEFAULT: }[ ]
{\\ \bf DATA TYPE: }str
\paragraph{Attributes}
 \begin{itemize}
\item {\bf shape}:
 The shape of the array
{\\ \bf DEFAULT: }(0,)
{\\ \bf DATA TYPE: }tuple
\end{itemize}
 
\item {\bf \hyperref[BEADS]{beads} }:
 Deals with path integral simulations.
\paragraph{Fields}
 \begin{itemize}
\item {\bf q}:
 The positions of the atoms, in the format [x1, y1, z1, x2, \ldots  ]
{\\ \bf DIMENSION: }length
{\\ \bf DEFAULT: }[ ]
{\\ \bf DATA TYPE: }float
\item {\bf p}:
 The momenta of the atoms, in the format [px1, py1, pz1, px2, \ldots  ]
{\\ \bf DIMENSION: }momentum
{\\ \bf DEFAULT: }[ ]
{\\ \bf DATA TYPE: }float
\item {\bf natoms}:
 The number of atoms
{\\ \bf DEFAULT: }0
{\\ \bf DATA TYPE: }int
\item {\bf start\_centroid}:
 An atoms object from which the centroid coordinates can be initialized
\item {\bf nbeads}:
 The number of beads
{\\ \bf DATA TYPE: }int
\item {\bf m}:
 The masses of the atoms, in the format [m1, m2, \ldots  ]
{\\ \bf DIMENSION: }mass
{\\ \bf DEFAULT: }[ ]
{\\ \bf DATA TYPE: }float
\item {\bf init\_temp}:
 The temperature at which the initial velocity distribution is taken, if applicable.
{\\ \bf DIMENSION: }temperature
{\\ \bf DEFAULT: }-1.0
{\\ \bf DATA TYPE: }float
\item {\bf names}:
 The names of the atoms, in the format [name1, name2, \ldots  ]
{\\ \bf DEFAULT: }[ ]
{\\ \bf DATA TYPE: }str
\end{itemize}
 
\item {\bf total\_steps}:
 The total number of steps that will be done.
{\\ \bf DEFAULT: }1000
{\\ \bf DATA TYPE: }int
\item {\bf prefix}:
 A string that will be the prefix for all the output file names.
{\\ \bf DEFAULT: }'prefix'
{\\ \bf DATA TYPE: }str
\item {\bf fd\_delta}:
 The parameter used in the finite difference differentiation in the calculation of the scaled path velocity estimator.
{\\ \bf DEFAULT: }0.0
{\\ \bf DATA TYPE: }float
\item {\bf \hyperref[CELL]{cell} }:
 Deals with the cell parameters, and stores their momenta in flexible cell calculations.
\paragraph{Attributes}
 \begin{itemize}
\item {\bf flexible}:
 Whether the cell parameters can change during the simulation.
{\\ \bf DEFAULT: }False
{\\ \bf DATA TYPE: }bool
\end{itemize}
 
\paragraph{Fields}
 \begin{itemize}
\item {\bf from\_file}:
 A file from which to take the cell parameters from.
{\\ \bf DEFAULT: }''
{\\ \bf DATA TYPE: }str
\item {\bf p}:
 The cell 'momenta' matrix, used in constant pressure simulations.
{\\ \bf DIMENSION: }momentum
{\\ \bf DEFAULT: }
      [[ 0.  0.  0.]
       [ 0.  0.  0.]
       [ 0.  0.  0.]]
{\\ \bf DATA TYPE: }float
\item {\bf init\_temp}:
 The temperature at which the initial velocity distribution is taken, if applicable.
{\\ \bf DIMENSION: }temperature
{\\ \bf DEFAULT: }-1.0
{\\ \bf DATA TYPE: }float
\item {\bf P}:
 The scalar cell 'momentum', used in constant pressure simulations.
{\\ \bf DIMENSION: }momentum
{\\ \bf DEFAULT: }0.0
{\\ \bf DATA TYPE: }float
\item {\bf file\_units}:
 The units in which the lengths in the configuration file are given.
{\\ \bf DEFAULT: }''
{\\ \bf OPTIONS: }'', 'nanometer', 'angstrom', 'atomic\_unit'
{\\ \bf DATA TYPE: }str
\item {\bf h}:
 The cell vector matrix
{\\ \bf DIMENSION: }length
{\\ \bf DEFAULT: }
      [[ 0.  0.  0.]
       [ 0.  0.  0.]
       [ 0.  0.  0.]]
{\\ \bf DATA TYPE: }float
\item {\bf h0}:
 The reference cell vector matrix. Defined as the unstressed equilibrium cell.
{\\ \bf DIMENSION: }length
{\\ \bf DEFAULT: }
      [[ 0.  0.  0.]
       [ 0.  0.  0.]
       [ 0.  0.  0.]]
{\\ \bf DATA TYPE: }float
\item {\bf m}:
 The 'mass' of the cell, used in constant pressure simulations.
{\\ \bf DIMENSION: }mass
{\\ \bf DEFAULT: }0.0
{\\ \bf DATA TYPE: }float
\end{itemize}
 
\item {\bf stride}:
 Dictionary holding the number of steps between printing the different kinds of files. The allowed keywords are ['checkpoint', 'properties', 'progress', 'trajectory', centroid']
{\\ \bf DEFAULT: }\{ \}
{\\ \bf DATA TYPE: }dict
\item {\bf traj\_format}:
 The file format for the output file. Allowed keywords are ['pdb', 'xyz'].
{\\ \bf DEFAULT: }'pdb'
{\\ \bf DATA TYPE: }str
\item {\bf \hyperref[RANDOM]{prng} }:
 Deals with the pseudo-random number generator.
\paragraph{Fields}
 \begin{itemize}
\item {\bf has\_gauss}:
 Determines whether there is a stored gaussian number or not. A value of 0 means there is none stored.
{\\ \bf DEFAULT: }0
{\\ \bf DATA TYPE: }int
\item {\bf state}:
 Gives the state vector for the random number generator. Avoid directly modifying this unless you are very familiar with the inner workings of the algorithm used.
{\\ \bf DEFAULT: }[ ]
{\\ \bf DATA TYPE: }uint64
\item {\bf seed}:
 This is the seed number used to generate the initial state of the random number generator.
{\\ \bf DEFAULT: }123456
{\\ \bf DATA TYPE: }int
\item {\bf set\_pos}:
 Gives the position in the state array that the random number generator is reading from.
{\\ \bf DEFAULT: }0
{\\ \bf DATA TYPE: }int
\item {\bf gauss}:
 The stored Gaussian number.
{\\ \bf DEFAULT: }0.0
{\\ \bf DATA TYPE: }float
\end{itemize}
 
\item {\bf \hyperref[ENSEMBLE]{ensemble} }:
 Holds all the information that is ensemble specific, such as the temperature and the external pressure, and the thermostats and barostats that control it.
\paragraph{Attributes}
 \begin{itemize}
\item {\bf type}:
 The ensemble that will be sampled during the simulation.
{\\ \bf DEFAULT: }'nve'
{\\ \bf OPTIONS: }'nve', 'nvt', 'npt', 'nst'
{\\ \bf DATA TYPE: }str
\end{itemize}
 
\paragraph{Fields}
 \begin{itemize}
\item {\bf stress}:
 The external stress.
{\\ \bf DIMENSION: }pressure
{\\ \bf DEFAULT: }
      [[ 1.  0.  0.]
       [ 0.  1.  0.]
       [ 0.  0.  1.]]
{\\ \bf DATA TYPE: }float
\item {\bf barostat}:
 Simulates an external pressure bath to keep the pressure or stress at the external values.
\item {\bf thermostat}:
 The thermostat for the atoms, keeps the atom velocity distribution at the correct temperature.
\item {\bf timestep}:
 The time step.
{\\ \bf DIMENSION: }time
{\\ \bf DEFAULT: }'1.0'
{\\ \bf DATA TYPE: }float
\item {\bf pressure}:
 The external pressure.
{\\ \bf DIMENSION: }pressure
{\\ \bf DEFAULT: }'1.0'
{\\ \bf DATA TYPE: }float
\item {\bf fixcom}:
 This describes whether the centre of mass of the particles is fixed.
{\\ \bf DEFAULT: }False
{\\ \bf DATA TYPE: }bool
\item {\bf temperature}:
 The temperature of the system.
{\\ \bf DIMENSION: }temperature
{\\ \bf DEFAULT: }1.0
{\\ \bf DATA TYPE: }float
\end{itemize}
 
\end{itemize}
 
