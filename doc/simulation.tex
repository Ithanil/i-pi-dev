\section{SIMULATION}
\label{SIMULATION}
This is the top level class that deals with the running of the simulation, including holding the simulation specific properties such as the time step and outputting the data.
\paragraph{Fields}
 \begin{itemize}
\item {\bf \hyperref[FORCES]{force} }:
 Deals with the assigning of jobs to different driver codes, and collecting the data.
\paragraph{Attributes}
 \begin{itemize}
\item {\bf type}:
 Specifies which kind of force object is created.
{\\ \bf DEFAULT: }'socket'
{\\ \bf OPTIONS: }'socket'
{\\ \bf DATA TYPE: }str
\end{itemize}
 
\item {\bf trajectories}:
 A list of the properties to print out the per-atom or per-bead trajectories of. Allowed values are ['positions', 'velocities', 'forces', 'kinetic\_cv', 'centroid'].
{\\ \bf DEFAULT: }[ ]
{\\ \bf DATA TYPE: }str
\paragraph{Attributes}
 \begin{itemize}
\item {\bf shape}:
 The shape of the array.
{\\ \bf DEFAULT: }(0,)
{\\ \bf DATA TYPE: }tuple
\end{itemize}
 
\item {\bf \hyperref[ATOMS]{atoms} }:
 Deals with classical simulations. Only needs to be specified if a classical simulation is required, and should be left blank otherwise.
\item {\bf step}:
 How many time steps have been done.
{\\ \bf DEFAULT: }0
{\\ \bf DATA TYPE: }int
\item {\bf initialize}:
 A dictionary giving the properties of the system that need to be initialized, and their initial values. The allowed keywords are ['velocities']. The initial value of 'velocities' corresponds to the temperature to initialise the velocity distribution from. If 0, then the sysytem temperature is used.
{\\ \bf DEFAULT: }\{ \}
{\\ \bf DATA TYPE: }dict
\item {\bf properties}:
 A list of the properties that will be printed in the properties output file. See the appropriate chapter in the manual for a full list of acceptable names.
{\\ \bf DEFAULT: }[ ]
{\\ \bf DATA TYPE: }str
\paragraph{Attributes}
 \begin{itemize}
\item {\bf shape}:
 The shape of the array.
{\\ \bf DEFAULT: }(0,)
{\\ \bf DATA TYPE: }tuple
\end{itemize}
 
\item {\bf \hyperref[BEADS]{beads} }:
 Deals with path integral simulations. Only needs to be specified if the atoms tag is not, but overwrites it otherwise.
\item {\bf total\_steps}:
 The total number of steps that will be done.
{\\ \bf DEFAULT: }1000
{\\ \bf DATA TYPE: }int
\item {\bf prefix}:
 A string that will be the prefix for all the output file names.
{\\ \bf DEFAULT: }'prefix'
{\\ \bf DATA TYPE: }str
\item {\bf fd\_delta}:
 The parameter used in the finite difference differentiation in the calculation of the scaled path velocity estimator. Defaults to 1e-5.
{\\ \bf DEFAULT: }0.0
{\\ \bf DATA TYPE: }float
\item {\bf \hyperref[CELL]{cell} }:
 Deals with the cell parameters, and stores their momenta in flexible cell calculations.
\paragraph{Attributes}
 \begin{itemize}
\item {\bf flexible}:
 Describes whether the simulation box shape can change, or just the volume.
{\\ \bf DEFAULT: }False
{\\ \bf DATA TYPE: }bool
\end{itemize}
 
\item {\bf stride}:
 Dictionary holding the number of steps between printing the different kinds of files. The allowed keywords are ['checkpoint', 'properties', 'progress', 'trajectory', centroid']. The default strides are {'checkpoint': 1000, 'properties': 10, 'progress': 100, 'centroid': 20, 'trajectory': 100}.
{\\ \bf DEFAULT: }\{ \}
{\\ \bf DATA TYPE: }dict
\item {\bf traj\_format}:
 The file format for the output file. Allowed keywords are ['pdb', 'xyz'].
{\\ \bf DEFAULT: }'pdb'
{\\ \bf DATA TYPE: }str
\item {\bf \hyperref[RANDOM]{prng} }:
 Deals with the pseudo-random number generator. It is not necessary to specify this tag.
\item {\bf \hyperref[ENSEMBLE]{ensemble} }:
 Holds all the information that is ensemble specific, such as the temperature and the external pressure, and the thermostats and barostats that control it.
\paragraph{Attributes}
 \begin{itemize}
\item {\bf type}:
 The ensemble that will be sampled during the simulation.
{\\ \bf DEFAULT: }'nve'
{\\ \bf OPTIONS: }'nve', 'nvt', 'npt', 'nst'
{\\ \bf DATA TYPE: }str
\end{itemize}
 
\end{itemize}
 
