\subsection{FORCES}
\label{FORCES}
Deals with the assigning of jobs to different driver codes, and collecting the data.
\paragraph{Attributes}
 \begin{itemize}
\item {\bf type}:
 Specifies which kind of force object is created
{\\ \bf DEFAULT: }'socket'
{\\ \bf OPTIONS: }'socket'
{\\ \bf DATA TYPE: }str
\end{itemize}
 
\paragraph{Fields}
 \begin{itemize}
\item {\bf \hyperref[INTERFACE]{interface} }:
 Specifies the parameters for the socket interface.
\paragraph{Attributes}
 \begin{itemize}
\item {\bf mode}:
 Specifies whether the driver interface will listen onto a internet socket [inet] or onto a unix socket [unix]
{\\ \bf DEFAULT: }'inet'
{\\ \bf OPTIONS: }'unix', 'inet'
{\\ \bf DATA TYPE: }str
\end{itemize}
 
\paragraph{Fields}
 \begin{itemize}
\item {\bf latency}:
 This gives the number of seconds between each check for new clients
{\\ \bf DEFAULT: }0.001
{\\ \bf DATA TYPE: }float
\item {\bf slots}:
 This gives the number of client codes that queue at any one time
{\\ \bf DEFAULT: }4
{\\ \bf DATA TYPE: }int
\item {\bf port}:
 This gives the port number that defines the socket
{\\ \bf DEFAULT: }31415
{\\ \bf DATA TYPE: }int
\item {\bf timeout}:
 This gives the number of seconds before assuming a calculation has died. If 0 there is no timeout.
{\\ \bf DEFAULT: }0.0
{\\ \bf DATA TYPE: }float
\item {\bf address}:
 This gives the server address that the socket will run on
{\\ \bf DEFAULT: }'localhost'
{\\ \bf DATA TYPE: }str
\end{itemize}
 
\item {\bf parameters}:
 deprecated dictionary of initialization parameters. May be removed in the future.
{\\ \bf DEFAULT: }\{ \}
{\\ \bf DATA TYPE: }dict
\end{itemize}
 
