\section{BEADS}
\label{BEADS}
Deals with path integral simulations.
\paragraph{Fields}
 \begin{itemize}
\item {\bf natoms}:
 The number of atoms.
{\\ \bf DEFAULT: }0
{\\ \bf DATA TYPE: }int
\item {\bf nbeads}:
 The number of beads.
{\\ \bf DATA TYPE: }int
\item {\bf m}:
 The masses of the atoms, in the format [m1, m2, \ldots  ].
{\\ \bf DIMENSION: }mass
{\\ \bf DEFAULT: }[ ]
{\\ \bf DATA TYPE: }float
\paragraph{Attributes}
 \begin{itemize}
\item {\bf shape}:
 The shape of the array.
{\\ \bf DEFAULT: }(0,)
{\\ \bf DATA TYPE: }tuple
\end{itemize}
 
\item {\bf q}:
 The positions of the beads. In an array of size [nbeads, 3*natoms].
{\\ \bf DIMENSION: }length
{\\ \bf DEFAULT: }[ ]
{\\ \bf DATA TYPE: }float
\paragraph{Attributes}
 \begin{itemize}
\item {\bf shape}:
 The shape of the array.
{\\ \bf DEFAULT: }(0,)
{\\ \bf DATA TYPE: }tuple
\end{itemize}
 
\item {\bf p}:
 The momenta of the beads. In an array of size [nbeads, 3*natoms].
{\\ \bf DIMENSION: }momentum
{\\ \bf DEFAULT: }[ ]
{\\ \bf DATA TYPE: }float
\paragraph{Attributes}
 \begin{itemize}
\item {\bf shape}:
 The shape of the array.
{\\ \bf DEFAULT: }(0,)
{\\ \bf DATA TYPE: }tuple
\end{itemize}
 
\item {\bf \hyperref[ATOMS]{start\_centroid} }:
 An atoms object from which the centroid coordinates can be initialized. Any parameters given here can be overwritten by specifying them explicitly.
\item {\bf names}:
 The names of the atoms, in the format [name1, name2, \ldots  ].
{\\ \bf DEFAULT: }[ ]
{\\ \bf DATA TYPE: }str
\paragraph{Attributes}
 \begin{itemize}
\item {\bf shape}:
 The shape of the array.
{\\ \bf DEFAULT: }(0,)
{\\ \bf DATA TYPE: }tuple
\end{itemize}
 
\end{itemize}
 
